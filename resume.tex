%!TEX TS-program = xelatex %!TEX encoding = UTF-8 Unicode
% Awesome CV LaTeX Template
%
% This template has been downloaded from:
% https://github.com/posquit0/Awesome-CV
%
% Author:
% Claud D. Park <posquit0.bj@gmail.com>
% http://www.posquit0.com
%
% Template license:
% CC BY-SA 4.0 (https://creativecommons.org/licenses/by-sa/4.0/)
%


%%%%%%%%%%%%%%%%%%%%%%%%%%%%%%%%%%%%%%
%     Configuration
%%%%%%%%%%%%%%%%%%%%%%%%%%%%%%%%%%%%%%
%%% Themes: Awesome-CV
\documentclass[]{awesome-cv}
\usepackage{textcomp}
%%% Override a directory location for fonts(default: 'fonts/')
\fontdir[fonts/]

%%% Configure a directory location for sections
\newcommand*{\sectiondir}{resume/}

%%% Override color
% Awesome Colors: awesome-emerald, awesome-skyblue, awesome-red, awesome-pink, awesome-orange
%                 awesome-nephritis, awesome-concrete, awesome-darknight
%% Color for highlight
% Define your custom color if you don't like awesome colors
%\colorlet{awesome}{awesome-red}
%\definecolor{awesome}{HTML}{CA63A8}
%% Colors for text
%\definecolor{darktext}{HTML}{414141}
%\definecolor{text}{HTML}{414141}
%\definecolor{graytext}{HTML}{414141}
%\definecolor{lighttext}{HTML}{414141}

%%% Override a separator for social informations in header(default: ' | ')
%\headersocialsep[\quad\textbar\quad]
    \begin{document}

%%%%%%%%%%%%%%%%%%%%%%%%%%%%%%%%%%%%%%
%     Profile
%%%%%%%%%%%%%%%%%%%%%%%%%%%%%%%%%%%%%%
\begin{center}
	\vspace{-5mm}
	\headerfirstnamestyle{Garrett} \headerlastnamestyle{Luu} \\
	\vspace{2mm}
  \faEnvelope\ \descriptionstyle{luu.garrett@gmail.com} |
  \faMobile\ \descriptionstyle{(714) 823-5110} |
  \faLink\ \descriptionstyle{garrettluu.com} |
  \faGithub\ \descriptionstyle{garrettluu} |
  \faLinkedinSquare\ \descriptionstyle{garrettluu}
\end{center}
%%%%%%%%%%%%%%%%%%%%%%%%%%%%%%%%%%%%%%
%     Education
%%%%%%%%%%%%%%%%%%%%%%%%%%%%%%%%%%%%%%
\vspace{-3mm}
\cvsection{Education}
\begin{cventries}
	\cventry
	{BS in Computer Science, Minor in Mathematics, 3.1 Overall GPA}
	{University of California, San Diego}
	{La Jolla, CA}
	{2018 – 2022}
  {\textbf{Coursework:} Advanced Data Structures, Algorithm Design/Analysis,
  Discrete Math \& Graph Theory, Software Engineering}
\end{cventries}

\vspace{-3mm}
%%%%%%%%%%%%%%%%%%%%%%%%%%%%%%%%%%%%%%
%     Experience
%%%%%%%%%%%%%%%%%%%%%%%%%%%%%%%%%%%%%%
\cvsection{Experience}
\begin{cventries}
	\cventry
	{Open Source Fellow}
	{Major League Hacking}
  {Remote}
	{June 2020 - Present}
	{\begin{cvitems}
    \item {Developing features, triaging issues, and reviewing pull requests for
      SheetJS, a \textbf{JavaScript} library for spreadsheets}
    \item {Refactored command-line interfaces to separate modules to increase
      maintainability and reduce package sizes}
    \item {Wrote additional unit tests using \textbf{Mocha} to verify behavior
      of new and existing features}
    \item {Developed and tested examples in \textbf{Azure} and \textbf{Firebase} to
      demonstrate file conversion features}
		\end{cvitems}}

	\vspace{-3mm}
	\cventry
	{Software Engineer}
	{IntElect}
	{La Jolla, CA}
	{February 2020 - Present}
	{\begin{cvitems}
    \item {Coordinated tasks using \textbf{Agile
      Project Management} to develop MVP to pitch to potential investors and
      startup incubators}
    \item {Developed back-end REST API
      using \textbf{Node.js}, \textbf{Express}, and \textbf{MongoDB}}
    \item {Designed \textbf{GraphQL} API to increase efficiency of data fetching
      and front-end/back-end communication}
		\end{cvitems}}

	\vspace{-3mm}
	\cventry
	{Web Development Intern}
	{StayLinked}
	{Irvine, CA}
	{August 2016}
	{\begin{cvitems}
    \item {Partnered with 2 other interns to develop web-app in 1 week using
      \textbf{HTML, CSS, jQuery} with \textbf{JavaScript}}
		\item {Developed a web-app for customizing toolbars for company’s software,
      used in production for over 250 corporate clients}
		\end{cvitems}}
\end{cventries}

\vspace{-5mm}
\cvsection{Activities}
\begin{cventries}
	\cventry
  {Co-founder and Co-president}
  {Association for Computing Machinery (ACM Hack)}
	{La Jolla, CA}
  {Nov 2019 – Present}
	{\begin{cvitems}
    \item {Co-founded student organization, along with 10 other students,
      dedicated to software engineering}
    \item {Collaborating with UCSD's ECE and CSE departments to offer certificates
      for \textbf{Python} workshops}
    \item {Developed and presented technical workshops on \textbf{Git}, \textbf{Unix},
        \textbf{React Native}, and \textbf{Firebase} for 100+ students}
		\end{cvitems}}

	\vspace{-3mm}
	\cventry
  {Software/Hardware Developer and  Mentor}
	{IEEE Quarterly Projects}
	{La Jolla, CA}
  {January 2019 – Present}
	{\begin{cvitems}
    \item {Developed heart-rate monitoring glove using \textbf{Arduino} and
      web application with \textbf{Node.js} to display health information}
    \item {Designed and assembled XY plotter device that draws images from svg
      files}
    \item {Collaborated in teams of 3-5 to design and build a project over the
      course of 9 weeks}
    \item {Mentored 3 teams of size 4-5 on design and development process}
		\end{cvitems}}

%  \vspace{-3mm}
%	\cventry
%	{Lead Programmer, President, Driver}
%	{FIRST Robotics Competition Team 4322}
%	{Orange, CA}
%	{August 2014 – June 2018}
%	{\begin{cvitems}
%		\item {Developed control systems and autonomous spline paths for competitive robots in
%      \textbf{Java}}
%    \item {Designed and wired control boards for robots using
%      \textbf{Fusion 360}}
%    \item {Coordinated tasks and activities in a team size of 10 using \textbf{Agile
%      Project Management}}
%		\end{cvitems}}


\end{cventries}

\vspace{-5mm}
\cvsection{Projects}
\begin{cventries}

	\cventry
  {Back-end Developer}
  {AI Flashcard Generator - HooHacks 2020 Project}
	{Express, React, Node.js}
	{github.com/gits-lit/hoo-hacks2020}
  {\begin{cvitems}
    \item{Collaborated in a team of 4 to design and build full-stack web
      application for AI generated flash cards}
    \item{Engineered back-end using \textbf{Express} and \textbf{Node.js} to
      pull data from Wikipedia to use for generating questions}
    \end{cvitems}}

  \vspace{-3mm}
	\cventry
  {Developer}
  {Carbon Footprint Extension - HackTech 2020 Project}
  {HTML, CSS, Javascript}
	{github.com/daniel-d-truong/web-waste}
  {\begin{cvitems}
    \item{\textbf{Chrome extension} that calculates carbon footprint of
      internet browsing}
    \item{Developed new features such as automatic pausing of background
      videos and preventing \\ double downloads to reduce carbon emissions of
      internet activity}
    \end{cvitems}}

  \vspace{-3mm}
	\cventry
  {Full Stack Developer}
	{Task Scheduling App - Hack at Home 2020 Project}
	{Flutter, Calendar API}
	{github.com/garrettluu/routine.ly}
  {\begin{cvitems}
    \item{Designed and developed mobile app to automatically schedule tasks with
      \textbf{Google Calendar API}}
    \item{Implemented user authentication and database features with
      \textbf{Firebase}}
  \end{cvitems}}

\end{cventries}

%\vspace{-5mm}
%\cvsection{Awards}
%\begin{cvhonors}
% \cvhonor
%	{UCSD Chancellor\textquotesingle{}s Scholarship}
%	{Awarded to high-achieving, low-income students}
%	{UC San Diego}
%	{March 2018}
%	\cvhonor
%	{FIRST Robotics Competition SpaceX Scholarship}
%	{Awarded to FIRST alumni based on software project involvement, one of 5
%    recipients}
%	{SpaceX}
%	{July 2018}
%	\cvhonor
%	{VPHS Mathematics Medallion Recipient}
%	{Recognizes exceptional students in mathematics, one of 4 recipients}
%	{Villa Park HS}
%	{June 2018}
%\end{cvhonors}

\vspace{-3mm}
\cvsection{Skills}
\begin{cventries}
  \vspace{-3mm}
	\cventry
	{}
	{\def\arraystretch{1.15}{\begin{tabular}{ l l l}
     Languages \hspace{5mm} & {\skill{ Java, C, C++, HTML, CSS, Javascript, Kotlin,
      Dart, Python, ARM Assembly}} \\
        Back-End  & {\skill{Node.js, Express, GraphQL, MongoDB, Firebase}} \\
        DevOps/Testing\hspace{5mm} & {\skill{ Git, Unix, GDB, Mocha, JUnit}} \\
            Frameworks & {\skill{React, React Native, Flutter, Electron, Next.js,
            jQuery}}
		\end{tabular}}}
	{}
	{}
	{}
\end{cventries}
\end{document}
